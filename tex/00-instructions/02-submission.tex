{\bf Submission Instructions}

{\bf Written Submission:}
All students must submit an electronic PDF version of the written questions. We
highly recommend typesetting your solutions via \LaTeX, though it is not
required. If you choose to hand write your responses, please make sure they are
well organized and legible when scanned. The source \LaTeX for all problem sets
is available on GitHub.

{\bf Coding Submission:}
You should modify the code in {\tt submission.py} between {\tt  START CODE
HERE} and {\tt END CODE HERE}, but you can add other helper functions outside
this block if you want. Do not make changes to files other than
{\tt submission.py}.  When finished, upload only {\tt submission.py} to
gradescope.

Your code will be evaluated on two types of test cases, {\bf basic} and
{\bf hidden}, which you can see in {\tt grader.py}. Basic tests, which are fully
provided to you, do not stress your code with large inputs or tricky corner
cases. Hidden tests are more complex and do stress your code.  The inputs of
hidden tests are provided in {\tt grader.py}, but the correct outputs are not.
To run the tests, you will need to have {\tt graderUtil.py} in the same
directory as your code and {\tt grader.py}. Then, you can run all the tests by
typing {\tt python grader.py}. This will tell you only whether you passed the
basic tests. On the hidden tests, the script will alert you if your code takes
too long or crashes, but does not say whether you got the correct output. You
can also run a single test (e.g., {\tt 3a-0-basic}) by typing {\tt python
grader.py 3a-0-basic}. We strongly encourage you to read and understand the
test cases, create your own test cases, and not just blindly run
{\tt grader.py}. You should make sure to (1) restrict yourself to only using
libraries included in the starter code, and (2) make sure your code runs without
errors.
